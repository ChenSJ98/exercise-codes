\documentclass[conference]{IEEEtran}

  \usepackage{booktabs}
  \usepackage{listing}
  \usepackage{amsmath}
  \usepackage{algorithm}
  \usepackage{array}
  \usepackage{url}
  \usepackage{cite}
  \usepackage{complexity}
\usepackage{algpseudocode}
% \usepackage{algorithm}
  \ifCLASSINFOpdf
  
  \else
  
  \fi
  
  \hyphenation{op-tical net-works semi-conduc-tor}
  
  
  \begin{document}
  
  \title{CS303 Project2: Genetics-based Heuristic Search Algorithm for Capacitated Arc Routing Problem}
  
  \author{\IEEEauthorblockN{Shijie Chen}
  \IEEEauthorblockA{Department of Computer Science and Engineering\\
  Southern University of Science and Technology\\
  Shenzhen, Guangdong, China\\
  Email: 11612028@mail.sustc.edu.cn}
  }
  
  \maketitle
  
  \begin{abstract}
  The Capacitated Arc Routing Problem (CARP) is one of the many arc routing problems and has been proven to be NP-hard. In this project, the author implemented a genetic-based heuristic search algorithm. The algorithm is able to obtain good result on both small and larger datasets.
  \end{abstract}
  \IEEEpeerreviewmaketitle
  
  \section{Preliminaries}

    The Capacitated Arc Routing Problem  is the problem of serving certain edges on a graph with a fleet of vehicles under a constraint on capacity\cite{Wohlk2008}. The objective is to minimize the total cost in order to service all the tasks.

    CARP has multiple applications in real life. For example, garbage collection and resource distribution. Problems that can be modeled as CARP often focus on serving edges rather than nodes.

	\begin{table}[H]
	\caption{representation}
	\centering
    \begin{tabular}{cccc}
    \toprule
    Name&Variable&Board&Pattern\\
    \midrule
    Black Stone&COLOR\_BLACK&-1&a\\
    White Stone&COlOR\_WHITE&1&b\\
    Empty Intersection&COLOR\_NONE&0&c\\
	\bottomrule
	\end{tabular}
	\label{table:1}
	\end{table}
  
\section{Methodology}

In this project, I used genetic algorithm as the main framework. Initial solutions is generated by a generalized path-scanning algorithm and Ulusoy split algorithm. Multiple mutation operators are designed to enhance search ability.
\subsection{Generalized Path-scanning}


\subsection{The Ulusoy Split Algorithm}



\subsection{Genetic Algorithm}
    \subsubsection{Framework}
    \subsubsection{Genetic Operators}

\section{Validation}
	 
\section{Discussion}


\section{Conclusion}

\section*{Acknowledgment}

The authors would like to thank the TAs for their hint in the Lab and maintaining a online runtime platform. 


\bibliographystyle{ieeetr}
\bibliography{ref}



% that's all folks
\end{document}


  