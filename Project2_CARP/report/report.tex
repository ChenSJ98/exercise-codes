\documentclass[conference]{IEEEtran}

  \usepackage{booktabs}
  \usepackage{listing}
  \usepackage{amsmath}
  \usepackage{algorithm}
  \usepackage{array}
  \usepackage{url}
  \usepackage{cite}
  \usepackage{complexity}
\usepackage{algpseudocode}
% \usepackage{algorithm}
  \ifCLASSINFOpdf
  
  \else
  
  \fi
  
  \hyphenation{op-tical net-works semi-conduc-tor}
  
  
  \begin{document}
  
  \title{CS303 Project2: Genetics-based Heuristic Search Algorithm for Capacitated Arc Routing Problem}
  
  \author{\IEEEauthorblockN{Shijie Chen}
  \IEEEauthorblockA{Department of Computer Science and Engineering\\
  Southern University of Science and Technology\\
  Shenzhen, Guangdong, China\\
  Email: 11612028@mail.sustc.edu.cn}
  }
  
  \maketitle
  
  \begin{abstract}
  The Capacitated Arc Routing Problem (CARP) is one of the many arc routing problems and has been proven to be NP-hard. In this project, the author implemented a genetic-based heuristic search algorithm. The algorithm is able to obtain good result on both small and larger datasets.
  \end{abstract}
  \IEEEpeerreviewmaketitle
  
  \section{Preliminaries}

    The Capacitated Arc Routing Problem  is the problem of serving certain edges on a graph with a fleet of vehicles under a constraint on capacity\cite{Wohlk2008}. The objective is to minimize the total cost in order to service all the tasks.

    CARP has multiple applications in real life. For example, garbage collection and resource distribution. Problems that can be modeled as CARP often focus on serving edges rather than nodes.

	
  
\section{Methodology}

In this project, I used genetic algorithm as the main framework. Initial solutions is generated by a generalized path-scanning algorithm and Ulusoy split algorithm. Multiple mutation operators are designed to enhance search ability.

\subsection{Notation}
The notations used in this report is shown in the table below.
    \begin{table}[H]
	\caption{representation}
	\centering
    \begin{tabular}{cccc}
    \toprule
    Name&Variable\\
    \midrule
    All tasks&tasks\\
    Tasks remain&undone\\
	\bottomrule
	\end{tabular}
	\label{table:1}
	\end{table}
\subsection{Genetic Algorithm}
    \subsubsection{Framework}
    A genetic based framework is used as the main search algorithm. 
    Initially, a population is created by heuristic approaches. In each iteration, apply certain genetic operations to the population. In this problem, since crossover between different solutions hardly makes sense, only mutation is used. After that, a selection based on the fitness value of individuals in the population is used to keep the size of population unchanged.
    \begin{algorithm}[H]
        \begin{algorithmic}[1]
            \State $population \gets initPopulation()$
            \State $size \gets len(population)$
            \While{ end condition not met}
                \State $offSpring \gets genOffspring(population)$
                \State $population \gets population + offSpring$
                \State $population.sort(key = cost)$
                \State $population \gets population[0:size]$
            \EndWhile
            
            \Return $population[0]$

        
        \end{algorithmic}
        \caption{Genetic Algorithm Framework}
    \end{algorithm}
    \subsubsection{Initial Population}
    
    The population is initiated by a Generalized Path Scanning algorithm, which generates a sequence of tasks to be done, and Ulusoy Split algorithm, which takes in a sequence of task and finds an optimal path in that sequence. Since Ulusoy Split algorithm only cares about the sequence of the task, load is not considered in the Generalized Path Scanning algorithm. Computational experiments also show that performance is better when path scanning ignores the capacity constraint.

    The DAG in Ulusoy Split is a graph with 2N nodes. Each edge in the graph has its starting point index less than its end point index. Each path in the graph with odd starting point and even end point denotes a route that finished all the tasks between the two ends. A path from node 1 to node 2N is a solution to the CARP problem.
        \begin{itemize}
            \item Ulusoy split
            Ulusoy Split works by converting the tasks to be done to a directed acyclic graph (DAG). Then obtain the optimal solution by searching for a shortest path in the DAG. Details of the algorithm can be found elsewhere. XXXXX
            \begin{algorithm}[H]
            \begin{algorithmic}[1]
                \Function{UlusoySplit}{tasks, depot, shortestPath, load}
                \State $DAG, incoming, outgoing \gets toDAG(tasks)$
                \For {$node \in DAG$}
                \State $minCost \gets inf$
                \State $bestEdge \gets minCost(incoming[node])$
                \State $node.bestPath \gets bestEdge.bestPath.append(node)$
                \State $node.cost \gets bestEdge.cost$
                \For{$edge \in outgoing[node]$}
                    \State $edge.cost \gets edge.cost+node.cost$
                    \State $edge.bestPath \gets node.bestPath$
                \EndFor
                \EndFor
                \State $x \gets 1$
                
                \Return $node[-1].cost, node[-1].bestPath$
                \EndFunction

                \Function {ToDAG}{tasks, depot, shortestPath, load}
                \State $construct\; the\; DAG$
                
                \Return DAG
                \EndFunction
            \end{algorithmic}
            \caption{Ulusoy Split}
            \end{algorithm}
            \item Generalized Path Scanning
            
            
        \end{itemize}

    \subsubsection{Genetic Operators}

\section{Validation}
	 
\section{Discussion}


\section{Conclusion}

\section*{Acknowledgment}

The authors would like to thank the TAs for their hint in the Lab and maintaining a online runtime platform. 


\bibliographystyle{ieeetr}
\bibliography{ref}



% that's all folks
\end{document}


  